\chapter*{Outline}
\markboth{Outline}{}
\addcontentsline{toc}{chapter}{Outline}
\label{chap:outline}
\thispagestyle{plain}

\chapterquote{I think it's much more interesting to live not knowing than to have answers which might be wrong.}
{Richard Feynman, 1918--1988}
\noindent
The development of Supersymmetry (\acrshort{susy}), a space-time symmetry relating classes of particles with different spin, has attracted the attention of physicists all around the world for the last few decades, and for good reason. It resolved a long-standing discomfort in interpreting the various energy scales of current fundamental physics theory - in a rather unique and extraordinary way. It promised new physics in the form of heavy `partners' to the Standard Model particles right around the corner, within reach of current-generation colliders, and certainly within that of the next. This appeal lead it to become one of the most studied candidates for physics beyond the Standard Model to date.

However, direct searches for these sparticles at colliders have so far turned up nothing. This leaves the prospect of supersymmetry as realized in nature an uncomfortable one. But why would the absence of supersymmetric partners at our current energies actually pose a threat to the validity of the theory one could ask? Although it in no way invalidates it, this is of course a legitimate question, and one that has troubled theorists and phenomenologists for decades. Can supersymmetry in its minimal form still resolve outstanding problems and require little to no fine-tuning of parameters to reproduce our observations? The idea of a `natural' supersymmetric theory may even today seem to be in conflict with what nature is telling us.

What preceded its development though, was the enormous success of the Standard Model of Particle Physics. Developed by both theorists and experimentalists alike, it can arguably be heralded as the greatest scientific achievement of the 20th century. From the success of Paul Dirac in describing relativistic spin-1/2 particles and the existence of anti-matter, to the discovery of the top quark at Fermilab in 1995 after being proposed by Kobayashi and Maskawa years earlier, the Standard Model almost seems impossible to fault. The recently discovered \cite{RN62,RN63} 'last piece' of the jigsaw puzzle, the Higgs boson, concluded this arduous century-long effort to describe the interactions of particles at the most elementary level. Though, whether it be the absence of a suitable dark matter candidate with the right density, or an explanation for the observed non-zero masses of the neutrinos, there was always a hint that it was incomplete - rather a low-energy description of a more fundamental theory.

In chapter \ref{chap:SMandSUSY}, we describe how supersymmetry emerged organically as a beyond-the-Standard Model candidate, for two important reasons: (1) Higgs loop-corrections sensitive to high-scale physics cancel each other exactly (to all orders in perturbation theory), and (2) it is the only extension to space-time symmetry that is compatible with the non-trivial $S$-matrix of \acrshort{qft}. Similarly, we highlight the many reasons why softly-broken SUSY arising at the TeV-scale is appealing, as it can provide unification of couplings at the high-scale, a weakly-interacting dark matter particle with approximately the correct density, whilst still maintaining exact cancellation of quadratic divergences. Specifically, we take a model-independent approach and introduce the Minimal Supersymmetric Standard Model (\acrshort{mssm}), parameterizing our ignorance of a \acrshort{uv}-complete theory with a full set of soft-breaking masses. This sets up an ideal phenomenological scenario for studying new physics in the context of colliders and cosmology.

In chapter \ref{chap:muong-2}, the MSSM is confronted with observations from experiment, where we concentrate on the challenge of explaining the muon anomalous magnetic moment, $(g-2)_{\mu}$. The $(g-2)_{\mu}$ is one of the most important low-energy observables for testing weak-scale supersymmetry, with exciting upcoming precision results that will push the bounds on minimal supersymmetry even further. Similarly, since we mainly focus on the $R$-Parity conserving case, the lightest supersymmetric particle (\acrshort{lsp}) forms an ideal dark matter candidate - where we study the predicted abundance and direct-detection rates in comparison with experimental values from various collaborations. Finally, we recast previous collider searches in channels sensitive to weak-scale supersymmetry and discuss the prospects for observations at future colliders with center of mass energies of $\sqrt{s}=100$ TeV.

In chapter \ref{chap:finetuning}, we explore the idea of `naturalness' in the MSSM. Naturalness as a guiding principle in the MSSM quantifies the sensitivity of the electroweak vacuum to changes in the fundamental parameters of the theory. We argue that regions of low fine-tuning of parameters may hint at physics beyond the MSSM, appearing at an arbitrary scale $\Lambda$ - and furthermore provide an example of low-fine tuning resulting from a theory defined at a higher scale through quasi-fixed Renormalization Group (RG) behavior. 

In chapter \ref{chap:wimpdecay}, we are motivated by the results of the previous two chapters showing that sections of the MSSM parameter space may be incompatible with observations of the present DM abundance in the universe. The existence of a temporary phase transition in the early universe that allows for decays of a dark matter candidate could be established, given that the symmetry stabilizing the DM is restored in the present (zero-temperature) phase. We also briefly discuss the prospect of accommodating this in the MSSM, by developing an $R$-parity violating vacuum in the sneutrino direction, and how we could further explore the effects of macroscopic conditions on the development of $R$-parity breaking phases in the early universe.

Chapter \ref{chap:nonlinearhiggs} presents an alternate description of electroweak symmetry breaking in the MSSM from non-linear realizations, still retaining the same model-independence and degrees of freedom present in the standard realization of the MSSM. We study the phenomenology, including the mass spectrum, predicted by the model.

We present our concluding remarks in chapter \ref{chap:conclusions} and supplementary material in the appendices.
