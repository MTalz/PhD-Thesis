%% Title
\titlepage[of The University of Sydney]{%
  A dissertation submitted for partial fulfillment of the requirements \\ for the degree of Doctor of Philosophy}

%% Abstract
\begin{abstract}%[\smaller \thetitle\\ \vspace*{1cm} \smaller {\theauthor}]
After the discovery of the Higgs boson at LHC \cite{RN62,RN63} and the subsequent measurements of its mass and couplings, the search for new physics has become more important than ever, with one of the most promising candidates being theories which exhibit supersymmetry (\acrshort{susy}). A space-time symmetry between fundamental integer and half-integer spin particles, SUSY proposes a plethora of new states which may be in reach of present-day collider technology, with even greater prospects for those of the future. As a consistent solution to the hierarchy problem, the unnaturally large quantum corrections to the bare Higgs boson mass from short-distance scale physics, this thesis explores anomalies from experiment left unexplained by the enormously successful Standard Model of particle physics, offered consistently even in minimal supersymmetric extensions.

With this in mind, we initially focus on the predictions from the Minimal Supersymmetric Standard Model (\acrshort{mssm}) without assumption of specific SUSY breaking mechanisms. Our MSSM phenomenology explores two of the most sensitive observables - the muon anomalous magnetic moment and dark matter. In conjunction with collider and other searches, we find a number of parameter regions still viable, though future 100 TeV collider searches may even be able to close the case on the MSSM explanation for these anomalies. Subsequently, we explore the idea of `naturalness' (or fine-tuning) in the MSSM. In light of current experimental limits on coloured sparticles, we believe naturalness considerations strongly hint the effectiveness of the MSSM up to new physics scales as low as 100 TeV.

Finally, we propose simple modification to minimal supersymmetry and its place in the early universe, without augmenting the gauge structure or particle content, in order to alleviate constraint on the allowable parameter space of the MSSM. The former focuses on the more comfortable accommodation of a 125 GeV Higgs boson mass within the framework of the MSSM through non-linear realization of electroweak gauge symmetry, whilst the latter accounts for a temporary cosmological dark matter (\acrshort{dm}) decay phase, avoiding the commonly encountered overabundance of DM in MSSM parameter regions.

Our results seem to yearn for higher center-of-mass energy collisions and more precise experimental observations in order to continue setting our targets and close in on supersymmetric models.
\end{abstract}

%% Declaration
\begin{declaration}
\noindent
This thesis presents new findings for the Minimal Supersymmetric Standard Model (\acrshort{mssm}), in a general model-independent fashion using a combination of analytical and numerical methods. It is the final piece of work towards completion of my doctorate in Theoretical Particle Physics at the University of Sydney.

Chapter \ref{chap:muong-2} is based on the published work in \cite{RN2} with corresponding conference notes in \cite{RN760}. I performed all major calculations for the low-energy observables, dark matter properties, and collider simulations and produced plots for the manuscript.

Chapter \ref{chap:finetuning} is based on the published work in \cite{RN803}. I performed all calculations and produced the plots as well as some contribution to the text in the manuscript.

Chapter \ref{chap:wimpdecay} is based on the published work in \cite{RN802}. I calculated some numerical results, derivations and plots.

Chapter \ref{chap:nonlinearhiggs} is based on the published work in \cite{RN1}. I calculated the potential contributions and mass spectrum for the manuscript.

Other research output during this doctorate that will not be fully discussed in this thesis are published in \cite{RN763,RN764,RN765}.
\newpage
\hspace{0pt}
\vfill
%\noindent
\textit{This certifies that the content of this thesis is entirely my own work, except where explicit reference is made to the work of others. This thesis has not been submitted as part of any other degree or qualification for this or any other university. I have acknowledged all sources of help in this thesis.}
  \vspace*{1cm}
  \begin{flushright}
  	\textit{Candidate's signature:}\hspace{0.5cm} \makebox[1.5in]{\hrulefill} \\
    \textit{Matthew Talia}\hspace{0.5cm}\, \\
    \vspace*{0.75cm}
    \textit{Supervisor's signature:}\hspace{0.5cm} \makebox[1.5in]{\hrulefill} \\
    \textit{Archil Kobakhidze}\hspace{0.3cm}\,
  \end{flushright}
\vfill
\hspace{0pt}
\end{declaration}

%% Acknowledgements
\begin{acknowledgements}
\noindent
Apologies in advance for this exhaustive, but necessary list. This thesis, amongst some other phenomena throughout time like neutron star mergers and actor Bill Paxton, is of course only made possible by all the fundamental particles and their interactions.

It is with immense gratitude that I must first acknowledge my advisor, Archil Kobakhidze for his support in all my endeavours without hesitation. With your guidance, I have learned to be humble in my efforts and to approach all science with a healthy dose of critical thought and skepticism. On a side note, Archil also promised that my dissertation would be accepted provided that I turned up to our regular soccer matches\footnote{Proper written citation for this statement pending.}.

I would also like to thank Michael Schmidt for his co-supervision and interesting discussions at many points during my doctoral studies. Special thanks to Lei Wu for help in developing my early interests in collider studies and computational work. And on that note, to Goncalo Borges for putting up with my continued (and at times, intensive) use of the computing cluster.

Thanks to those who supported me in my travels around Spain, northern Italy and the UK to present my work, particularly Zurab Berezhiani, Dmitri Sorokin and Francesco Coradeschi for invited talks. I should also mention the ARC Center of Excellence for Particle Physics at the Terascale (CoEPP) for supporting this research from the beginning.

It has been a privilege to work with some excellent minds in Room 342 over the years, and I could see myself benefiting greatly from some fantastic theoretical discussions, especially Neil Barrie, Sundy Arunasalam, Cyril Lagger and anyone else I have forgotten. Most of all, during my time in Sydney, I got to make some great friends (Lachlan and Abed) whom I will never forget. And of course, everyone back home in Melbourne, some of whom I've known almost all my life.

To Mum and Dad and all the family back home, thank you for all the support from such a distance. It is because of you that I made it here, safe and sound. To Dean, you made it to the finish line first, I just kind-of followed along hoping that I had the potential to do the same. And of course, to my partner-in-crime Sophie who was there every (and I mean every) step of the way. Fine, I'll say thanks to our cat, Newton. Is that better now?

To close this especially important section, it is often a thought of mine that we seldom remember to thank ourselves for overcoming our own trials and tribulations. Along with the special help of friends, family, my partner, members of the department, and other important people along the way, I was able to slowly navigate the treacherous dark tunnel that is mental health recovery and emerge from the other side - something I thought I would never be capable of.

\end{acknowledgements}

% Preface
%\begin{preface}
%
%\end{preface}

%%Acronyms
\clearpage
 
\printglossary[type=\acronymtype,nonumberlist]

%\printglossary[type=\acronymtype]
 
%\printglossary

% Given a set of numbers, there are elementary methods to compute 
% its \acrlong{gcd}, which is abbreviated \acrshort{gcd}. This 
% process is similar to that used for the \acrfull{lcm}.

%% ToC
\tableofcontents

%% Strictly optional!
\frontquote{ \centering
I'm not afraid of death. I'm an old physicist - I'm afraid of time.
 }%
  {Dr. Brand, from \textit{Interstellar (2014) \cite{RN759}.}}
%% I don't want a page number on the following blank page either.
\thispagestyle{empty}

%%Outline
\chapter*{Outline}
\markboth{Outline}{}
\addcontentsline{toc}{chapter}{Outline}
\label{chap:outline}
\thispagestyle{plain}

\chapterquote{I think it's much more interesting to live not knowing than to have answers which might be wrong.}
{Richard Feynman, 1918--1988}
\noindent
The development of Supersymmetry (\acrshort{susy}), a space-time symmetry relating classes of particles with different spin, has attracted the attention of physicists all around the world for the last few decades, and for good reason. It resolved a long-standing discomfort in interpreting the various energy scales of current fundamental physics theory - in a rather unique and extraordinary way. It promised new physics in the form of heavy `partners' to the Standard Model particles right around the corner, within reach of current-generation colliders, and certainly within that of the next. This appeal lead it to become one of the most studied candidates for physics beyond the Standard Model to date.

However, direct searches for these sparticles at colliders have so far turned up nothing. This leaves the prospect of supersymmetry as realized in nature an uncomfortable one. But why would the absence of supersymmetric partners at our current energies actually pose a threat to the validity of the theory one could ask? Although it in no way invalidates it, this is of course a legitimate question, and one that has troubled theorists and phenomenologists for decades. Can supersymmetry in its minimal form still resolve outstanding problems and require little to no fine-tuning of parameters to reproduce our observations? The idea of a `natural' supersymmetric theory may even today seem to be in conflict with what nature is telling us.

What preceded its development though, was the enormous success of the Standard Model of Particle Physics. Developed by both theorists and experimentalists alike, it can arguably be heralded as the greatest scientific achievement of the 20th century. From the success of Paul Dirac in describing relativistic spin-1/2 particles and the existence of anti-matter, to the discovery of the top quark at Fermilab in 1995 after being proposed by Kobayashi and Maskawa years earlier, the Standard Model almost seems impossible to fault. The recently discovered \cite{RN62,RN63} 'last piece' of the jigsaw puzzle, the Higgs boson, concluded this arduous century-long effort to describe the interactions of particles at the most elementary level. Though, whether it be the absence of a suitable dark matter candidate with the right density, or an explanation for the observed non-zero masses of the neutrinos, there was always a hint that it was incomplete - rather a low-energy description of a more fundamental theory.

In chapter \ref{chap:SMandSUSY}, we describe how supersymmetry emerged organically as a beyond-the-Standard Model candidate, for two important reasons: (1) Higgs loop-corrections sensitive to high-scale physics cancel each other exactly (to all orders in perturbation theory), and (2) it is the only extension to space-time symmetry that is compatible with the non-trivial $S$-matrix of \acrshort{qft}. Similarly, we highlight the many reasons why softly-broken SUSY arising at the TeV-scale is appealing, as it can provide unification of couplings at the high-scale, a weakly-interacting dark matter particle with approximately the correct density, whilst still maintaining exact cancellation of quadratic divergences. Specifically, we take a model-independent approach and introduce the Minimal Supersymmetric Standard Model (\acrshort{mssm}), parameterizing our ignorance of a \acrshort{uv}-complete theory with a full set of soft-breaking masses. This sets up an ideal phenomenological scenario for studying new physics in the context of colliders and cosmology.

In chapter \ref{chap:muong-2}, the MSSM is confronted with observations from experiment, where we concentrate on the challenge of explaining the muon anomalous magnetic moment, $(g-2)_{\mu}$. The $(g-2)_{\mu}$ is one of the most important low-energy observables for testing weak-scale supersymmetry, with exciting upcoming precision results that will push the bounds on minimal supersymmetry even further. Similarly, since we mainly focus on the $R$-Parity conserving case, the lightest supersymmetric particle (\acrshort{lsp}) forms an ideal dark matter candidate - where we study the predicted abundance and direct-detection rates in comparison with experimental values from various collaborations. Finally, we recast previous collider searches in channels sensitive to weak-scale supersymmetry and discuss the prospects for observations at future colliders with center of mass energies of $\sqrt{s}=100$ TeV.

In chapter \ref{chap:finetuning}, we explore the idea of `naturalness' in the MSSM. Naturalness as a guiding principle in the MSSM quantifies the sensitivity of the electroweak vacuum to changes in the fundamental parameters of the theory. We argue that regions of low fine-tuning of parameters may hint at physics beyond the MSSM, appearing at an arbitrary scale $\Lambda$ - and furthermore provide an example of low-fine tuning resulting from a theory defined at a higher scale through quasi-fixed Renormalization Group (RG) behavior. 

In chapter \ref{chap:wimpdecay}, we are motivated by the results of the previous two chapters showing that sections of the MSSM parameter space may be incompatible with observations of the present DM abundance in the universe. The existence of a temporary phase transition in the early universe that allows for decays of a dark matter candidate could be established, given that the symmetry stabilizing the DM is restored in the present (zero-temperature) phase. We also briefly discuss the prospect of accommodating this in the MSSM, by developing an $R$-parity violating vacuum in the sneutrino direction, and how we could further explore the effects of macroscopic conditions on the development of $R$-parity breaking phases in the early universe.

Chapter \ref{chap:nonlinearhiggs} presents an alternate description of electroweak symmetry breaking in the MSSM from non-linear realizations, still retaining the same model-independence and degrees of freedom present in the standard realization of the MSSM. We study the phenomenology, including the mass spectrum, predicted by the model.

We present our concluding remarks in chapter \ref{chap:conclusions} and supplementary material in the appendices.
