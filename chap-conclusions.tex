\chapter{Conclusions}
\label{chap:conclusions}

Beginning with the simplest of supersymmetric models, the MSSM, it is apparent that the major outstanding challenges that face the Standard Model of Particle Physics can be addressed quite elegantly. In particular, one can aptly resolve the hierarchy problem by stabilizing the electroweak scale from large quantum corrections. But perhaps one of the most exciting consequence is the prediction of new physics right around the corner. In fact, the case is made for explanations of outstanding low-energy observables, the most interesting of these being the muon anomalous magnetic moment and weakly-coupled dark matter. The intimate connection between them makes for an excellent probe of the potential of supersymmetric models.

We investigate this in detail in chapter \ref{chap:muong-2}, with emphasis on the potential for discovery at current and future collider searches. Because of the increasing effort from the experimental and theoretical sides, especially with upcoming high precision measurements at BNL \cite{RN116}, we focused our sights on the muon $(g-2)_{\mu}$, finding important constraint on the electroweakinos and sleptons. Moreover, with constraints coming from dark matter experiments and observations, we clearly see a preference for specific arrangements of parameters in the theory, namely the composition of neutralino DM. It is probable that a significant portion of the space maintaining the explanation for these observations may be probed in the future, tightening the grip on the possibility of the realization of supersymmetry in nature.

In chapter \ref{chap:finetuning}, we explored the idea of naturalness in the context of the MSSM. As the experiments from the LHC push the masses of supersymmetric particles (particularly coloured sparticles like gluinos and stops) higher and higher, the amount of fine-tuning required to reproduce the weak-scale becomes stronger and stronger. We take the attitude that this failure seems to suggest physics beyond the MSSM. For the MSSM as an effective theory though, certain parameters are almost insensitive to their values in the ultraviolet (the so called 'quasi fixed-points' \cite{RN747,RN748}), leading to significant reduction in fine-tuning. This is in agreement with our expectations compared with the dynamics of the theory described by the Renormalization Group Equations (\acrshort{rge}s).

In this dissertation, however, we have also considered a number of novel modifications, not only to the realization of the theory, but also to the macroscopic conditions in early universe cosmology - leading to relaxation on the allowable parameter space that would otherwise limit the range of observables that it has the potential to explain. The latter we consider in chapter \ref{chap:wimpdecay}, augmenting the standard cosmological model of dark matter with an additional phase where decays are permitted. We find that the MSSM may permit such a phase through sneutrino condensation, spontaneously violating $R$-parity, which would usually render the lightest neutralino stable. In general, we find that this scenario may lead to a significant reduction in the abundance of dark matter in the early universe, a feature of some models, especially the MSSM neutralino with a large bino fraction.

Chapter \ref{chap:nonlinearhiggs} considered an alternate arrangement of the electroweak sector of the MSSM, one in which the Higgs is integrated out in an $\SUgroup{2} \times \Ugroup{1}_Y$ singlet superfield, $S$. This unique configuration gave rise to non-standard linear and cubic terms in $S$, modifying the phenomenology of the Higgs bosons. As well as accommodating the lighter Higgs mass through extra tree-level contributions, alleviating unnecessary fine-tuning for the electroweak vacuum. We also found that electroweak symmetry could even be broken in the supersymmetric limit.

The overall message conveyed in this thesis is two-fold. Firstly, we demonstrated the ability for minimal supersymmetry to address inconsistencies in our current knowledge of fundamental physics from both the experimental and theoretical side. Not only can these be facilitated within the MSSM itself, but they also strongly suggest that SUSY could even appear within the crosshair of next-generation collider technology. Secondly, the inability for minimal supersymmetry to accommodate them comfortably, coupled with the absence of sparticles at colliders, seems to hint at new physics beyond-the-MSSM, or simple modification to the microscopic and/or macroscopic properties of the theory that continue to be studied further. With these considerations, SUSY could still be a very real possibility just out of our present reach, but as elusive as it may seem, it certainly remains that supersymmetry bears a rather unique place in our quest to discover new and exciting physics.

