\usepackage{xspace}
\usepackage{tikz}
\usetikzlibrary{patterns,intersections,calc,decorations.markings}
\usepackage{graphicx}
\usepackage{pdfpages}
\usepackage{slashed}
%\usepackage{morefloats,subfig,afterpage}  % Check this later
%\usepackage{mathrsfs} % script font
\usepackage{caption}
\usepackage{amsmath}
\usepackage{nccmath}
\usepackage{amstext}
\usepackage{amssymb}
\usepackage{axodraw2}
\usepackage{color}
\usepackage{verbatim}
\usepackage{threeparttable}
\usepackage{multirow}
\usepackage{float}
\usepackage{enumitem}
\DeclareOldFontCommand{\bf}{\normalfont\bfseries}{\mathbf}

\usepackage{mdwlist}

\usepackage{braket}

\usepackage{blindtext} %For referencing Footnotes

\usepackage[acronym]{glossaries}

\usepackage{subcaption}
\captionsetup{compatibility=false}

%\usepackage[autostyle]{csquotes}

\makeglossaries
 
\newacronym{gcd}{GCD}{Greatest Common Divisor}
\newacronym{lcm}{LCM}{Least Common Multiple}
\newacronym{dreg}{DREG}{Dimensional Regularization}
\newacronym{dred}{DRED}{Dimensional Reduction}
\newacronym{ms}{MS}{Minimal Subtraction}
\newacronym{fcnc}{FCNC}{Flavor-Changing Neutral Current}
\newacronym{ckm}{CKM}{Cabibbo-Kobayashi-Maskawa}
\newacronym{mfv}{MFV}{Minimal Flavor Violating}
\newacronym{pmsb}{PMSB}{Planck-Scale Mediated Supersymmetry Breaking}
\newacronym{gmsb}{GMSB}{Gauge Mediated Supersymmetry Breaking}
\newacronym{amsb}{AMSB}{Anomaly Mediated Supersymmetry Breaking}
\newacronym{uv}{UV}{Ultraviolet}
\newacronym{ir}{IR}{Infrared}
\newacronym{mssm}{MSSM}{Minimal Supersymmetric Standard Model}
\newacronym{nmssm}{NMSSM}{Next-to-Minimal Supersymmetric Standard Model}
\newacronym{msugra}{mSUGRA}{Minimal Supergravity}
\newacronym{sm}{SM}{Standard Model}
\newacronym{bsm}{BSM}{Beyond the Standard Model}
\newacronym{dm}{DM}{Dark Matter}
\newacronym{cdm}{CDM}{Cold Dark Matter}
\newacronym{susy}{SUSY}{Supersymmetry}
\newacronym{lsp}{LSP}{Lightest Supersymmetric Particle}
\newacronym{nlsp}{NLSP}{Next-to-Lightest Supersymmetric Particle}
\newacronym{ewsb}{EWSB}{Electroweak Symmetry Breaking}
\newacronym{vev}{VEV}{Vacuum Expectation Value}
\newacronym{cmb}{CMB}{Cosmic Microwave Background}
\newacronym{edm}{EDM}{Electric Dipole Moment}
\newacronym{mdm}{MDM}{Magnetic Dipole Moment}
\newacronym{qed}{QED}{Quantum Electrodynamics}
\newacronym{qcd}{QCD}{Quantum Chromodynamics}
\newacronym{lhc}{LHC}{Large Hadron Collider}
\newacronym{wimp}{WIMP}{Weakly Interacting Massive Particle}
\newacronym{rge}{RGE}{Renormalization Group Equation}
\newacronym{eft}{EFT}{Effective Field Theory}
\newacronym{qft}{QFT}{Quantum Field Theory}
\newacronym{bnl}{BNL}{Brookhaven National Laboratory}
\newacronym{sf}{SF}{Superfield}
\newacronym{lep}{LEP}{Large Electron-Positron Collider}
\newacronym{hllhc}{HL-LHC}{High-Luminosity Large Hadron Collider}
\newacronym{slha}{SLHA}{SUSY Les Houches Accord \cite{RN762,RN761}}
\newacronym{vbf}{VBF}{Vector Boson Fusion}
\newacronym{atlas}{ATLAS}{A Toroidal LHC Apparatus}
\newacronym{cms}{CMS}{Compact Muon Solenoid}
\newacronym{bch}{BCH}{Baker-Campbell-Hausdorff}
\newacronym{np}{NP}{New Physics}
\newacronym{gut}{GUT}{Grand Unified Theory}
\newacronym{qfp}{QFP}{Quasi-Fixed Point}
\newacronym{ccwz}{CCWZ}{Callan, Coleman, Wess and Zumino}
\newacronym{cp}{CP}{Charge-Parity}
\newacronym{2hdm}{2HDM}{2 Higgs-Doublet Model}
\newacronym{rwesb}{RWESB}{Radiative Electroweak Symmetry Breaking}

%% Using Babel allows other languages to be used and mixed-in easily
%\usepackage[ngerman,english]{babel}
\usepackage[english]{babel}
\selectlanguage{english}

%% Citation system tweaks
\usepackage{cite}
% \let\@OldCite\cite
% \renewcommand{\cite}[1]{\mbox{\!\!\!\@OldCite{#1}}}

%% Maths
% TODO: rework or eliminate maybemath
\usepackage{abmath}
\DeclareRobustCommand{\mymath}[1]{\ensuremath{\maybebmsf{#1}}}
% \DeclareRobustCommand{\parenths}[1]{\mymath{\left({#1}\right)}\xspace}
% \DeclareRobustCommand{\braces}[1]{\mymath{\left\{{#1}\right\}}\xspace}
% \DeclareRobustCommand{\angles}[1]{\mymath{\left\langle{#1}\right\rangle}\xspace}
% \DeclareRobustCommand{\sqbracs}[1]{\mymath{\left[{#1}\right]}\xspace}
% \DeclareRobustCommand{\mods}[1]{\mymath{\left\lvert{#1}\right\rvert}\xspace}
% \DeclareRobustCommand{\modsq}[1]{\mymath{\mods{#1}^2}\xspace}
% \DeclareRobustCommand{\dblmods}[1]{\mymath{\left\lVert{#1}\right\rVert}\xspace}
% \DeclareRobustCommand{\expOf}[1]{\mymath{\exp{\!\parenths{#1}}}\xspace}
% \DeclareRobustCommand{\eexp}[1]{\mymath{e^{#1}}\xspace}
% \DeclareRobustCommand{\plusquad}{\mymath{\oplus}\xspace}
% \DeclareRobustCommand{\logOf}[1]{\mymath{\log\!\parenths{#1}}\xspace}
% \DeclareRobustCommand{\lnOf}[1]{\mymath{\ln\!\parenths{#1}}\xspace}
% \DeclareRobustCommand{\ofOrder}[1]{\mymath{\mathcal{O}\parenths{#1}}\xspace}
% \DeclareRobustCommand{\SOgroup}[1]{\mymath{\mathup{SO}\parenths{#1}}\xspace}
% \DeclareRobustCommand{\SUgroup}[1]{\mymath{\mathup{SU}\parenths{#1}}\xspace}
% \DeclareRobustCommand{\Ugroup}[1]{\mymath{\mathup{U}\parenths{#1}}\xspace}
% \DeclareRobustCommand{\I}[1]{\mymath{\mathrm{i}}\xspace}
% \DeclareRobustCommand{\colvector}[1]{\mymath{\begin{pmatrix}#1\end{pmatrix}}\xspace}
\DeclareRobustCommand{\Rate}{\mymath{\Gamma}\xspace}
\DeclareRobustCommand{\RateOf}[1]{\mymath{\Gamma}\parenths{#1}\xspace}

%% High-energy physics stuff
\usepackage{abhep}
\usepackage{hepnames}
\usepackage{hepunits}
\DeclareRobustCommand{\arXivCode}[1]{arXiv:#1}
\DeclareRobustCommand{\CP}{\ensuremath{\mathcal{CP}}\xspace}
\DeclareRobustCommand{\CPviolation}{\CP-violation\xspace}
\DeclareRobustCommand{\CPv}{\CPviolation}
\DeclareRobustCommand{\LHCb}{LHCb\xspace}
\DeclareRobustCommand{\LHC}{LHC\xspace}
\DeclareRobustCommand{\LEP}{LEP\xspace}
\DeclareRobustCommand{\CERN}{CERN\xspace}
\DeclareRobustCommand{\bphysics}{\Pbottom-physics\xspace}
\DeclareRobustCommand{\bhadron}{\Pbottom-hadron\xspace}
\DeclareRobustCommand{\Bmeson}{\PB-meson\xspace}
\DeclareRobustCommand{\bbaryon}{\Pbottom-baryon\xspace}
\DeclareRobustCommand{\Bdecay}{\PB-decay\xspace}
\DeclareRobustCommand{\bdecay}{\Pbottom-decay\xspace}
\DeclareRobustCommand{\BToKPi}{\HepProcess{ \PB \to \PK \Ppi }\xspace}
\DeclareRobustCommand{\BToPiPi}{\HepProcess{ \PB \to \Ppi \Ppi }\xspace}
\DeclareRobustCommand{\BToKK}{\HepProcess{ \PB \to \PK \PK }\xspace}
\DeclareRobustCommand{\BToRhoPi}{\HepProcess{ \PB \to \Prho \Ppi }\xspace}
\DeclareRobustCommand{\BToRhoRho}{\HepProcess{ \PB \to \Prho \Prho }\xspace}
\DeclareRobustCommand{\X}{\thesismath{X}\xspace}
\DeclareRobustCommand{\Xbar}{\thesismath{\overline{X}}\xspace}
\DeclareRobustCommand{\Xzero}{\HepGenParticle{X}{}{0}\xspace}
\DeclareRobustCommand{\Xzerobar}{\HepGenAntiParticle{X}{}{0}\xspace}
\DeclareRobustCommand{\epluseminus}{\Ppositron\!\Pelectron\xspace}
\DeclareRobustCommand{\protonproton}{\Pproton\APantiproton\xspace}
